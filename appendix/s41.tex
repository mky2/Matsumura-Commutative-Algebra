\documentclass[../main]{subfiles}
\begin{document}

\section{Krull Rings and Marot's Theorem}\label{sec:41}

\newparagraph Let $A$ be an integral domain and put $P=\{\ideal{p} \in \Spec(A) \mid\Ht\ideal{p}=1\}$. We call $A$ a \defemph{Krull ring}\index{Krull ring} if
\begin{enumerate}[label = (\arabic*)]
    \item $A_{\ideal{p}}$ is a principal valuation ring for all $\ideal{p} \in P$, and
    \item every non-zero principal ideal $aA$ is the intersection of a finite number of primary ideals of height $1$.
\end{enumerate}

A normal Noetherian domain is a Krull ring by Th.\ref{thm:037} and Th.\ref{thm:038} We will give a sufficient condition for the converse to hold. First we list a few elementary properties of Krull rings. Let $A$ be a Krull ring with quotient field $K$.
\begin{enumerate}[label=\Roman*)]
    \item Let $a, b \in A$, $\, a \neq 0$, $\, x=b / a$. By (2) we have $aA =\ideal q_1 \cap \cdots \cap \ideal q_r$, $\,\ideal q_i=a_{\ideal p_i} \cap A_1$, $\ideal p_i \in p$. Therefore $x \in A\iff b \in \ideal q_i$ for all $i$ $\iff b \in A_{\ideal p}$ for all $\ideal p \in P$. Hence $A=\bigcap_{\ideal p} A_{\ideal p}$. Moreover, if $0 \neq x \in K$, then $x$ is a unit in $A_{\ideal p}$ for all but a finite number of $\ideal p \in P$.
    \item By (1) each primary ideal $\ideal q$ of height $1$ is a symbolic power of its radical. Therefore every principal ideal $aA \neq 0$ is of the form \[a A=\ideal p_1^{(n_1)} \cap \cdots \cap \ideal p_r^{(n_r)}\for{\ideal p_i \in P}.\]
    \item If $\ideal p \in P$, let $v_\ideal p(\,\cdot\,)$ denote the normalized valuation associated to $A_\ideal p$ (i.e. if $\ideal p A_\ideal p=t_\ideal p A_\ideal p$ then $v_\ideal p(x)=n$ means $xA_\ideal p=t_\ideal p^nA_\ideal p$). Then for each $0 \neq x \in K$ there exists at most a finite number of $\ideal p \in P$ with $v_\ideal p(x) \neq 0$. If $a \in A$ we can write $aA=\bigcap {\ideal p}^{(v_{\ideal p}(a))}$.
    \item If $\dim A=1$ then $A$ is Noetherian. Indeed, let $I$ be an ideal. If $I \neq(0)$ pick $a \in I$, $a \neq 0$. It suffices to prove that $I/aA$ is a finite module. Writing aA as in II), we can embed $A/a$ in $A/\ideal p_1^{(n_1)}\oplus\cdots\oplus A/\ideal p_r^{(n_r)}$. But if $\ideal p \in P$ then $\ideal p$ is maximal and $A / \ideal p^{(n)}$ is a module of finite length. This proves our assertion. An integral domain in which every non-zero ideal is uniquely represented as the product of a finite number of prime ideals is called a \defemph{Dedekind domain}\index{Dedekind domain}. It is well known that an integral domain is Dedekind iff it is normal, Noetherian and of dimension $\leqslant 1$. Therefore Krull domains of dimension $\leqslant 1$ are nothing but Dedekind domains.
    \item Suppose we are given $\ideal p_1, \ldots, \ideal p_r \in P$ and $e_1, \ldots, e_r \in \bZ$. Then there exists $x \in K$ satisfying \[v_{\ideal p_1}(x)=e_i\quad(1 \leqslant i \leqslant r), \quad v_{\ideal p}(x) \geqslant 0\text{ for all other }\ideal p \in P.\]
\end{enumerate}

\begin{proof}
Take $y_1 \in \ideal p_1-(\ideal p_1^{(2)} \cup \ideal p_2 \cup \ldots \cup \ideal p_r)$. Then $v_i(y_1)=\delta_{i1}\for{1 \leqslant i \leqslant r}$. Similarly, take $y_j \in A$ such that $v_i(y_j)=\delta_i j\for{1 \leqslant i \leqslant r}$ and put $y=\prod y_i^{e_i}$. Put $P'=P-\{\ideal p_1, \ldots, \ideal p_r\}$. There exists at most a finite number of $\ideal p \in P'$ such that $v_{\ideal p}(y)<0$; denote them by $p_1', \ldots, p_s'$. Take $t_j \in \ideal p_j'-(\ideal p_1 \cup \ldots \cup \ideal p_r)$ for $1 \leqslant j \leqslant s$, and put $x=y(t_1 \ldots t_s)^n$ with $n$ sufficiently large. Then $x$ satisfies our requirement.
\end{proof}

\begin{partheorem}[Y.Mort - J.Nishimura]\label{thm:104}
Let $A$ be a Krull ring and $P$ be as before. If $A / \ideal p$ is Noetherian for every $\ideal p \in P$, then $A$ is Noetherian.
\end{partheorem}

\begin{proof}
We will prove that $A / \ideal p^{(n)}$ is Noetherian (as a ring, or what is the same, as an $A$-module) for every $\ideal p \in P$ and for every $n>0$. Since a finite sum of Noetherian modules is again Noetherian, and since any submodule of a Noetherian module is Noetherian by definition, it then follows that $A$ is Noetherian as in the proof of IV).

Using V) for $e_1=-1$ we can find $x \in \Phi A$ such that $v_{\ideal p}(x)=1$, $v_{\ideal q}(x) \leqslant 0$ for all $\ideal q \in P-\{\ideal p\}$. Put $B=A[x]$. If $y \in \ideal p$ then $y / x \in A$, hence $\ideal p \subseteq x B \cap A$. Conversely, since $B \subseteq A_{\ideal p}$ and $x B \subseteq \ideal p A_{\ideal p}$ we have $\ideal p \supseteq x B \cap A$. Therefore $\ideal p=x B \cap A$, and $B=A+x B$, hence $B / x B \cong A / \ideal p$. Since $x^nB / x^{n+1} B \cong B / x B$ for all $n$, it is clear that $B / x^n B$ is Noetherian for all $n$. But \[x^nB \cap A \subseteq x^nA_{\ideal p} \cap A=\ideal p^{(n)}\] and $B / x^nB$ is generated by the images of $1, x, \ldots, x^{n-1}$ over $A/(x^n B \cap A)$ By Eakin's theorem if $A/(x^nB \cap A)$ is a Noetherian ring, of which $A / \ideal p^{(n)}$ is a homomorphic image. Therefore $A / \ideal p^{(n)}$ is Noetherian, as wanted.
\end{proof}

\begin{theorem*}[Mort-Nagata Integral Closure Theorem]
Let $A$ be a Noetherian domain with quotient field $k$, and $L$ be a finite algebraic extension of $K$. Then the integral closure $A'$ of $A$ in $L$ is Krull ring. If $P' \in \Spec A'$ and $P=P' \cap A$, then $[\kappa(P'):\kappa(P)]<\infty$. If $P \in \Spec A$, there exists only a finite number of prime ideals of $A'$ lying over $P$.
\end{theorem*}

For the proof we refer to \cite{nagata1975local} or to \cite{fossum2012divisor}. (In fact they consider the case $L=K$, but the general case is easily reduced to this case by enlarging $A$ a little.) They use the structure theorem of complete local rings. Recently, J. Nishimura (\cite{nishimura1976note}) and J. Querr\'e (\cite{querre1977on}) gave different proofs of the first assertion which do not use the structure theorem. 

\begin{partheorem*}[Krull-Akizuki]
If $\dim A=1$ in the preceding theorem, every ring between $L$ and $A$ is Noetherian.
\end{partheorem*}

For the proof see \cite[ch.7]{bourbaki1998commutative} or \cite{matijevic1976maximal}.

\begin{theorem}\label{thm:105}
If $\dim A=2$ in the Mori-Nagata theorem, then $A'$ is Noetherian.
\end{theorem}

\begin{proof}
Let $P'$ be a prime ideal of height $1$ in $A'$. Then $A' / P'$ is integral over $A / P$, where $P=P' \cap A$, $\,[\kappa(P'): \kappa(P)]$ is finite and $\dim A / P=1$. Therefore $A' / P'$ is Noetherian by the Krull-Akizuki theorem, hence $A'$ is Noetherian by Th.\ref{thm:104}.
\end{proof}

\begin{partheorem}[J. Marot]\label{thm:106}
Let $A$ be a Noetherian ring and $I$ an ideal of $A$. Suppose that $A$ is complete and separated in the $I$-adic topology and that $A/I$ is a Nagata ring. Then $A$ is a Nagata ring.
\end{partheorem}

\begin{proof}
We have to prove that $A / \ideal p$ is N-2 for all $\ideal p \in \Spec(A)$. Assume the contrary. Then there exists a maximal element $\ideal p_0\in\{\ideal p\mid A /\ideal p\text{ is not N-2}\}$. The hypotheses on $A$ are inherited by all homomorphic images of $A$ (note that\newline $I \subseteq\rad(A)$). Replacing $A$ by $A / \ideal p_0$, we may therefore assume that $A$ is a Noetherian domain, that $A / \ideal p$ is N-2 if $(0) \neq \ideal p \in \Spec(A)$ and that $A$ is not N-2 (hence $I \neq (0)$). Let $K$ be the quotient field of $A$, $L$ be a finite algebraic extension of $K$ and $B$ be the integral closure of $A$ in $L$. If $(0) \neq P \in \Spec(B)$ and $P \cap A=\ideal p$, then $\ideal p \neq(0)$ and $[\kappa(P): \kappa(\ideal p)]<\infty$. Therefore $B / P$ is finite over $A / \ideal p$ by the N-2 property of $A / \ideal p$, and so $ B / P$ is Noetherian. Therefore $B$ is Noetherian by Th.\ref{thm:104}. Let $R$ be the radical of $I B$ and let $R=P_1 \cap \ldots \cap P_r$ be its prime decomposition. Put $\ideal p_i=P_i \cap A$. Then $\ideal p_i \supseteq I \neq(0)$, hence $A / \ideal p_i$ is N-2 and $B / P_i$ is finite over $A / \ideal p_i$ for all $i$. Since $B / R$ can be embedded in $B / P_1 \oplus \cdots \oplus B / P_r$ and since $A$ is Noetherian, $B / R$ is a finite $A$-module. Since $B$ is Noetherian, $R^n / R^{n+1}$ is a finite module over $B / R$, hence also over $A$, for all $n$. Using the exact sequence \[0 \longrightarrow R^n / R^{n+1} \longrightarrow B / R^{n+1} \longrightarrow B / R^n \longrightarrow 0\] we see inductively that $B / R^n$ is finite over $A$ for all $n$. Since $R^n\subseteq IB$ for $n$ sufficiently large, $B/IB$ is also finite over $A$. Since $B$ is Noetherian and \newline $IB \subseteq \rad(B)$, $B$ is separated in the $I$-adic topology. Therefore $B$ is finite over $A$ by Lemma \ref{lem:28.01}. This proves that $A$ is N-2, contrary to our assumption.
\end{proof}
\end{document}