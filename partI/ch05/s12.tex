\documentclass[../main]{subfiles}
\begin{document}

\section{Dimension}\label{sec:12}

\newparagraph
Let $A$ be a ring, $A \neq 0$. A finite sequence of $n+1$ prime ideals \newline $\ideal{p}_0 \supset \ideal{p}_1 \supset \cdots\supset\ideal{p}_n$ is called a \defemph{prime chain}\index{prime chain} of length $n$. If $\ideal{p} \in \Spec(A)$, the supremum of the lengths of the prime chains with $\ideal{p}=\ideal{p}_0$ is called the \defemph{height} of $\ideal{p}$ and denoted by $\Ht(\ideal{p})$. Thus $\Ht(\ideal{p})=0$ means that $\ideal{p}$ is a minimal prime of $A$.

Let $I$ be a proper ideal of $A$. We define the \defemph{height}\index{height} of $I$ to be the minimum of the heights of the prime ideals containing $I$: $\Ht(I)=\inf\{\Ht(\ideal{p}) \mid \ideal{p} \supseteq I\}$.

The \defemph{dimension}\index{dimension} of $A$ is defined to be the supremum of the heights of the prime ideals in $A$: \[\dim(A)=\sup\{\Ht(\ideal{p}) \mid \ideal{p} \in \Spec(A)\}.\] It is also called the \defemph{Krull dimension}\index{Krull dimension} of $A$. If $\dim(A)$ is finite then it is equal to the length of the longest prime chains in $A$. For example, a principal ideal domain has dimension one.

It follows from the definition that \[\Ht(\ideal{p})=\dim(A_{\ideal{p}}) \for{\ideal{p} \in \Spec(A)},\] and that, for any ideal $I$ of $A$, \[\dim(A/I)+\Ht(I)\leqslant\dim(A).\]

\newparagraph Let $M \neq 0$ be an $A$-module. We define the \defemph{dimension}\index{dimension} of $M$ by \[\dim(M)=\dim(A/\Ann(M)).\] (When $M=0$ we put $\dim(M)=-1$.) Under the assumption that $A$ is Noetherian and $M \neq 0$ is finite over $A$, the following conditions are equivalent:
\begin{enumerate}[label=(\arabic*)]
    \item $M$ is an $A$-module of finite length,
    \item the ring $A/\Ann(M)$ is Artinian,
    \item $\dim(M)=0$
\end{enumerate}

In fact, (3) $\iff$ (2) $\implies$ (1) is obvious by \ref{2.C}. Let us prove\newline (1) $\implies$ (3). We suppose $\length(M)$ is finite, and replacing $A$ by $A/\Ann(M)$ we assume that $\Ann(M)=(0)$. If $\dim(A)>0$, take a minimal prime $\ideal{p}$ of $A$ which is not maximal. Since $M$ is finite over $A$ and since $\Ann(M)=(0)$, we easily see that $M_{\ideal{p}} \neq 0$. Hence $\ideal{p}$ is a minimal member of $\Supp(M)$, so that $\ideal{p} \in \Ass(M)$. Then $M$ contains a submodule isomorphic to $A / \ideal{p}$, and since $\dim(A / \ideal{p})>0$ we have $\length(A / \ideal{p})=\infty$, contradiction. Therefore $\dim(A)$ $(=\dim(M))$ $=0$.

\newparagraph Let $A$ be a Noetherian semi-local ring, and $\ideal{m}=\rad(A)$. An ideal $I$ is called an \defemph{ideal of definition}\index{ideal of definition} of $A$ if $\ideal{m}^\nu \subseteq I \subseteq \ideal{m}$ some $\nu>0$. This is equivalent to saying that \[I \subseteq \ideal{m}, \text{ and }A / I\text{ is Artinian}.\] Let $I$ be an ideal of definition and $M$ a finite $A$-module. Put \begin{align*}
    A^*&=\gr^I(A)=\bigoplus I^n / I^{n+1},\\
    \text{and }M^*&=\gr^I(M)=\bigoplus I^n M/I^{n+1} M.
\end{align*}
Let $I=Ax_1+\cdots+Ax_r$. Then the graded ring $A^*$ is a homomorphic image of $B=(A / I)[X_1, \ldots, X_r]$, and $M^*$ is a finite, graded $A^*$-module, Therefore $F_{M^*}(n)=\length(I^n M / I^{n+1} M)$ is a polynomial in $n$, of degree $\leqslant r-1$, for $n\gg 0$. It follows that the function \[\chi(M, I;n)\underset{\text{def}}{=}\length(M / I^n M)=\sum_{j=0}^{n-1} F_{M^*}(j)\] is also a polynomial in $n$, of degree $\leqslant r$, for $n \gg 0$. The polynomial which represents $\chi(M, I ; n)$ for $n \gg 0$ is called the \defemph{Hilbert polynomial} of $M$ with respect to $I$. If $J$ is another ideal of definition of $A$, then $J^s \subseteq I$ for some $s>0$, so that we have $\chi(M, I ; n) \leqslant \chi(M, J ; s n)$. Thus, if $\chi(M, I ; n)=a_dn^d+\cdots+a_0$ and $\chi(M, J ; n)=b_{d'}n^{d'}+\cdots+b_0$, then $d\leqslant d'$. By symmetry we get $d=d'$. Thus the degree $d$ of the Hilbert polynomial is independent of the choice of $I$. We denote it by $\hilbertd(M)$. Remember that, if there exists an ideal of definition of $A$ generated by $r$ elements, then $\hilbertd(M) \leqslant r$.

\begin{parproposition}
Let $A$ be a Noetherian semi-local ring, $I$ an ideal of definition of $A$ and \[0 \longrightarrow M' \longrightarrow M \longrightarrow M'' \longrightarrow 0\] an exact sequence of finite $A$-modules. Then $\hilbertd(M)=\max(\hilbertd(M'), \hilbertd(M''))$. Moreover, $\chi(M, I ; n)-\chi(M', I ; n)-\chi(M'', I ; n)$ is a polynomial of degree $<\hilbertd(M')$ for $n \gg 0$.
\end{parproposition}

\begin{proof}
Since \[\length(M'' / I^n M'')=\length(M / M'+I^n M) \leqslant \length(M / I^n M),\] we get \newline $\hilbertd(M'') \leqslant \hilbertd(M)$. Furthermore, \[\begin{aligned}\chi(M, I ; n)-\chi(M'', I ; n)&=\length(M / I^nM)-\length(M / M'+I^n M)\\&=\length(M'+I^n M / I^n M)\\&=\length(M' / M' \cap I^n M),\end{aligned}\] and there exists $r>0$ such that $M' \cap I^n M \subseteq I^{n-r} M'$ for $n>r$ by Artin-Rees. Thus \[\length(M' / I^n M') \geqslant \length(M' / M' \cap I^n M) \geqslant \length(M' / I^{n-r} M').\] This means that $\chi(M, I ; n)-\chi(M'', I ; n)$ and $\chi(M', I ; n)$ have the same degree and the same leading term.
\end{proof}

\begin{parlemma}\label{lem:12.01}
Let $A$ be a Noetherian semi-local ring. Then $\hilbertd(A) \geqslant \dim(A)$
\end{parlemma}

\begin{proof}
Induction on $\hilbertd(A)$. If $\hilbertd(A)=0$ then $\ideal{m}^\nu=\ideal{m}^{\nu+1}=\ldots$ for some $\nu>0$. By the intersection theorem (\ref{11.D} Cor.\ref{cor:11.01}), this implies $\ideal{m}^\nu=(0)$. Hence $\length(A)$ is finite and $\dim(A)=0$. Suppose $\hilbertd(A)>0$. As the case $\dim(A)=0$ is trivial, we assume $\dim(A)>0$. Let \[\ideal{p}_0\supset\cdots\supseteq\ideal{p}_{e-1} \supset \ideal{p}_e=\ideal{p}\] be a prime chain of length $e>0$, and take an element $x \in \ideal{p}_{e-1}$ such that $x \notin \ideal{p}$. Then $\dim(A /(xA+\ideal{p})) \geqslant e-1$. Applying the preceding proposition to the exact sequence \[0 \longrightarrow A / \ideal{p} \varrightarrow{x} A / \ideal{p} \longrightarrow A / (xA+\ideal{p}) \longrightarrow 0\] we have $\hilbertd(A / (xA+\ideal{p}))<\hilbertd(A / \ideal{p}) \leqslant \hilbertd(A)$. Thus, by induction hypothesis we get \[e-1 \leqslant \dim(A / (xA+\ideal{p})) \leqslant \hilbertd(A / (xA+\ideal{p}))<\hilbertd(A).\] Hence $e\leqslant \hilbertd(A)$, therefore $\dim(A)\leqslant\hilbertd(A)$.
\end{proof}

\begin{remark}
The lemma shows that the dimension of $A$ is finite. When $A$ is an arbitrary Noetherian ring and $\ideal{p}$ is a prime ideal, we have $\Ht(\ideal{p})=\dim(A_{\ideal{p}})$ so that $\Ht(\ideal{p})$ is finite. (This was first proved by Krull by a different method.) Thus the descending chain condition holds for prime ideals in a Noetherian ring. On the other hand, there are Noetherian rings with infinite dimension.
\end{remark} 

\begin{parlemma}\label{lem:12.02}
Let $A$ be a Noetherian semi-local ring, $M \neq 0$ a finite $A$-module, and $x \in\rad (A)$. Then \[\hilbertd(M) \geqslant\hilbertd(M / x M) \geqslant \hilbertd(M)-1.\]
\end{parlemma}

\begin{proof}
Let $I$ be an ideal of definition containing $x$. Then \[\chi(M/xM, I;n)=\length(M/(xM+I^nM))=\length(M/I^nM)-\length((x M+I^n M) / I^n M)\] and \[(x M+I^n M) / I^n M \cong x M / (x M \cap I^n M) \cong M /(I^n M: x)\] and $I^{n-1} M \subseteq(I^n M: x)$, therefore
\begin{align*}
\chi(M/xM,I;n)&\geqslant\length(M/I^nM)-\length(M/I^{n-1}M)\\
&=\chi(M, I ; n)-\chi(M, I ; n-1).
\end{align*}
It follows that $\hilbertd(M / x M) \geqslant \hilbertd(M)-1$.
\end{proof}

\begin{parlemma}\label{lem:12.03}
Let $A$ and $M$ be as above, and let $\dim(M)=r$. Then there exist $r$ elements $x_1, \ldots, x_r$ of $\rad(A)$ such that \[\length(M / (x_1 M+\cdots+x_r M))<\infty.\]
\end{parlemma}

\begin{proof}
Let $I$ be an ideal of definition of $A$. When $r=0$ we have $\length(M)<\infty$ and the assertion holds. Suppose $r>0$ and let $\ideal{p}_1, \ldots, \ideal{p}_t$ be those minimal prime over-ideals of $\Ann(M)$ which satisfy $\dim(A / \ideal{p}_i)=r$. Then no maximal ideals are contained in any $\ideal{p}_i$, hence $\rad(A) \nsubseteq \ideal{p}_i\for{1 \leqslant i \leqslant t}$. Thus by \ref{1.B} there exists $x_1 \in \rad(A)$ which is not contained in any $\ideal{p}_i$. Then $\dim(M / x_1 M) \leqslant r-1$, and the assertion follows by induction on $\dim(M)$.
\end{proof}

\begin{partheorem}\label{thm:017}
Let $A$ be a Noetherian semi-local ring, $\ideal{m}=\rad(A)$ and $M \neq 0$ a finite $A$-module. Then $\hilbertd(M)=\dim M=$ the smallest integer $r$ such that there exist elements $x_1,\ldots, x_r$ of $\ideal{m}$ satisfying $\length(M / (x_1 M+\cdots+x_r M))<\infty$.
\end{partheorem}

\begin{proof}
If $\length(M / (x_1 M+\cdots+x_r M))<\infty$ we have $\hilbertd(M)\leqslant r$ by Lemma \ref{lem:12.02}. When $r$ is the smallest possible we have $r \leqslant \dim(M)$ by Lemma \ref{lem:12.03}. It remains to prove $\dim(M) \leqslant \hilbertd(M)$. Take a sequence of submodules $M=M_1\supset M_2 \supset \cdots \supset M_{k+1}=(0)$ such that \[M_i / M_{i+1} \cong A / \ideal{p}_i, \ideal{p}_i \in \Spec(A).\] Then $\ideal{p}_i \supseteq\Ann(M)$ and $\Ass(M)\subseteq\{\ideal{p}_1, \ldots, \ideal{p}_k\}$. Since $\Supp(M) \neq V(\Ann(M))$ all the minimal over-ideals of $\Ann(M)$ are in $\Ass(M)$ (hence also in $\{\ideal{p}_1, \ldots, \ideal{p}_k\})$ by \ref{7.D}. Therefore
\begin{align*}
    \hilbertd(M)&=\max\hilbertd(A/\ideal{p}_i) & &\text{by \ref{12.D}}\\
    &\geqslant\max\dim(A/\ideal{p}_i) & &\text{by Lemma \ref{lem:12.01}}\\
    &=\dim(A/\Ann(M))=\dim(M), & &
\end{align*}
which completes the proof.
\end{proof} 

\begin{partheorem}\label{thm:018}
Let $A$ be a Noetherian ring and $I=(a_1, \ldots, a_r)$ be an ideal generated by $r$ elements. Then any minimal prime over-ideal $\ideal{p}$ of $I$ has height $\leqslant r$. In particular, $\Ht(I)\leqslant r$.
\end{partheorem}

\begin{proof}
Since $\ideal{p}A_{\ideal{p}}$ is the only prime ideal of $A_{\ideal{p}}$ containing $IA_{\ideal{p}}$, the ring \[A_{\ideal{p}}/IA_{\ideal{p}}=A_{\ideal{p}} /(a_1 A_{\ideal{p}}+\cdots+a_r A_{\ideal{p}})\] is Artinian. Therefore $\Ht(\ideal{p})=\dim(A_{\ideal{p}}) \leqslant r$ by Th.\ref{thm:017}.
\end{proof} 

\newparagraph Let $(A, \ideal{m}, k)$ be a Noetherian local ring of dimension $d$. In this case, an ideal of definition of $A$ and a primary ideal belonging to $\ideal{m}$ are the same thing. We know (Th.\ref{thm:017}) that no ideals of definition are generated by less than $d$ elements, and that there are ideals of definition generated by exactly $d$ elements. If $(x_1, \ldots, x_d)$ is an ideal of definition then we say that $\{x_1, \ldots, x_d\}$ is a \defemph{system of parameters}\index{system of parameters} of $A$. If there exists a system of parameters generating the maximal ideal $\ideal{m}$, then we say that $A$ is a \defemph{regular local ring}\index{regular!\indexline local ring} and we call such a system of parameters a \defemph{regular system of parameters}\index{regular!\indexline system of parameters}. Since the number of elements of a minimal basis of $\ideal{m}$ is equal to rank $\ideal{m} / \ideal{m}^2$, we have in general \[\dim(A) \leqslant\rank_k\ideal{m} / \ideal{m}^2,\] and the equality holds iff $A$ is regular.

\begin{parproposition}\label{pro:12.02}
Let $(A, \ideal{m})$ be a Noetherian local ring and $x_1, \ldots, x_d$ a system of parameters of $A$. Then \[\dim(A/(x_1, \ldots, x_i))=d-i=\dim(A)-i\] for each $1 \leqslant i \leqslant d$.
\end{parproposition}

\begin{proof}
Put $\overline{A}=A/(x_1, \ldots, x_i)$. Then $\dim(\overline{A}) \leqslant d-i$ since $\overline{x}_{i+1}, \ldots, \overline{x}_d$ generate an ideal of definition of $\overline{A}$. On the other hand, if $\dim(\overline{A})=p$ and if $\overline{y}_1, \ldots, \overline{y}_p$ is a system of parameters of $\overline{A}$, then $x_1, \ldots, x_i, y_1, \ldots, y_p$ generate an ideal of definition of $A$ so that $p+i \geqslant d$, that is, $p \geqslant d-i$.
\end{proof}

\end{document}